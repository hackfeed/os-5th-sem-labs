\section*{Цель работы}

Знакомство со средством дизассемблирования Sourcer, получение дизассемблированного кода ядра операционной системы Windows на примере обработчика прерывания INT 8h в virtual mode – специальном режиме защищенного режима (32-разрядный режим работы), который эмулирует реальный режим работы  вычислительной системы на базе процессоров Intel.

\section*{Задание}

Используя Sourcer получить дизассемблированный код обработчика аппаратного прерывания от системного таймера INT 8h.

На основе полученного кода составить алгоритм работы обработчика INT 8h.

\section*{Листинг кода}

\begin{lstinputlisting}[
	caption={Обработчик INT 8h},
	label={lst:int8h},
	style={asm},
	linerange={1-76}
]{../src/INT8H.ASM}
\end{lstinputlisting}

\begin{lstinputlisting}[
	caption={Процедура sub\_1},
	label={lst:sub_1},
	style={asm},
	linerange={77-102}
]{../src/INT8H.ASM}
\end{lstinputlisting}

\clearpage

\section*{Схема алгоритма}

\img{220mm}{int8h_1}{Схема обработчика прерываний INT 8h}

\img{220mm}{int8h_2}{Схема обработчика прерываний INT 8h}

\img{220mm}{sub1}{Схема процедуры sub\_1}

\clearpage

\section*{Вывод}

Функции обработчика прерывания INT 8h в DOS:

\begin{itemize}
	\item Увеличивает текущее значение четырехбайтовой переменной, располагающейся в области данных BIOS по адресу 0000:046Ch. По этому адресу располагается счетчик тиков таймера. Если этот счетчик переполняется (после 24 часов с момента запуска таймера), в ячейку 0000:0470h заносится 1.
	\item Контроль за работой двигателей моторчика дисковода. Если после последнего обращения к НГМД прошло более 2 секунд, обработчик прерывания выключает двигатель. Ячейка с адресом 0000:0440h содержит время, оставшееся до выключения двигателя. Это время постоянно уменьшается обработчиком прерывания таймера. Когда оно становится равно 0, двигатель НГМД отключается.
	\item Вызов пользовательского прерывания 1Ch. Его стандартный обработчик состоит из одной команды IRET. Во время выполнения прерывания INT 1Ch все аппаратные прерывания запрещены.
\end{itemize}
